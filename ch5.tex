\section{Chapter V}
\translation{
  Quae cum ita sint, Catilina, perge, quo coepisti, egredere aliquando ex urbe; patent portae; proficiscere. Nimium diu te imperatorem tua illa Manliana castra desiderant. Educ tecum etiam omnes tuos, si minus, quam plurimos; purga urbem. Magno me metu liberabis, dum modo inter me atque te murus intersit. Nobiscum versari iam diutius non potes; non feram, non patiar, non sinam.

  11. Magna dis inmortalibus habenda est atque huic ipsi Iovi Statori, antiquissimo custodi huius urbis, gratia, quod hanc tam taetram, tam horribilem tamque infestam rei publicae pestem totiens iam effugimus. Non est saepius in uno homine summa salus periclitanda rei publicae. Quamdiu mihi consuli designato, Catilina, insidiatus es, non publico me praesidio, sed privata diligentia defendi. Cum proximis comitiis consularibus me consulem in campo et competitores tuos interficere voluisti, compressi conatus tuos nefarios amicorum praesidio et copiis nullo tumultu publice concitato; denique, quotienscumque me petisti, per me tibi obstiti, quamquam videbam perniciem meam cum magna calamitate rei publicae esse coniunctam.

  12. Nunc iam aperte rem publicam universam petis, templa deorum inmortalium, tecta urbis, vitam omnium civium, Italiam totam ad exitium et vastitatem vocas. Quare, quoniam id, quod est primum, et quod huius imperii disciplinaeque maiorum proprium est, facere nondum audeo, faciam id, quod est ad severitatem lenius et ad communem salutem utilius. Nam si te interfici iussero, residebit in re publica reliqua coniuratorum manus; sin tu, quod te iam dudum hortor, exieris, exhaurietur ex urbe tuorum comitum magna et perniciosa sentina rei publicae. 

13. Quid est, Catilina? num dubitas id me imperante facere, quod iam tua sponte faciebas? Exire ex urbe iubet consul hostem. Interrogas me: num in exilium? Non iubeo, sed, si me consulis, suadeo.}{
  Since these things are thus, Catiline, go on to where you have begun going, leave the city now at last; the gates are open: set out! For too long has that Manlian camp of yours been longing for you as a general. Lead away with you even all your men, or if not all, then as many as possible, cleanse the city. You will free me from great fear, provided only that there is a wall between me and you. You cannot live with us for any longer: I will not bear it, I will not endure it, I will not allow it.

  Great thanks must be given to the immortal gods and to this Jupiter Stator himself, the most ancient guardian of this city, because we have already so often escaped this scourge, so foul, so dreadful and so hostile to the republic. The utmost safety of the republic must not be so often put at risk by one man. All the time you plotted against me, as consul elect, Catiline, I defended myself not with a public garrison but with private vigilance. When you wished to kill me as consul, and your competitors, at the last consular elections in the Campus Martii, I suppressed your nefarious attempts with the guard and resources of friends, without any official mobilisation of troops; besides, when as often as you attacked me, I thwarted you by myself, though I saw that my ruin was joined with great disaster for the republic.

  Now you are already openly attacking the entire republic; you call the temples of the immortal gods, the buildings of the city, the lives of all the citizens, the whole of Italy to destruction and devastation. Therefore, seeing as I do not yet dare to do that which is first and foremost, and what is keeping with this supreme authority of the Senate and best traditions of the ancestors, I will do that which is more lenient with regard to strictness and more useful with regard to common safety. For if I order you to be killed, your band of conspirators will remain behind in the remaining republic; but if you leave, which I have been encouraging you to do for a long time, the great scum of your companions, dangerous to the republic, will be drained from the city.

  What is it, Catiline? Surely you do not hesitate to do it at my command, which you were already doing of your own accord. The consul is ordering the enemy of the state to leave the city. You ask me: surely not into exile? I am not ordering you, but, if you consult me, I urge you.
}
