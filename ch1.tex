\section{Chapter I}
\translation{
  1. \note{Quo}{interrogative} \note{usque}{adverb modifies quo} tandem \note{abutere}{future deponent second person singular of abutor, syncopated form of abuteris. quo + indicative = direct question}, Catilina, \note{patientia}{ablative after abutere} nostra? \note{quam}{interrogative} diu etiam furor iste tuus nos \note{eludet}{eludo eludere elusi elusum}? quem ad finem \note{sese}{accusative direct object of iactabit} effrenata \note{iactabit}{quem + indicative = direct question} \note{audacia}{subject of iactabit}? \note{Nihilne}{anaphora of nihil from lines 3--6} te nocturnum praesidium \note{Palatii}{genitive of classification/objective}, nihil urbis vigiliae, nihil timor \note{populi}{subjective genitive}, nihil concursus bonorum \note{omnium}{possessive genitive}, nihil hic munitissimus \note{habendi}{gerundive, modifies senatus, genitive of classification} senatus locus, nihil horum ora voltusque moverunt? \note{Patere}{accusative (consilia) + infinitive = indirect statement depending on sentis} tua \note{consilia}{accusative + infinitive patere = subject of indirect statement depending on sentis} non sentis? constrictam iam \note{horum omnium}{genitive of possession} scientia \note{teneri}{accusative (coniurationem) + infinitive = indirect statement dependent on vides} \note{coniurationem}{accusative + infinitive (teneri) = indirect statement dependent on vides} tuam non vides? \note{Quid}{all interrogatives} proxima, quid superiore nocte \note{egeris}{quid + subjunctive = indirect question dependent on ignorare}, ubi \note{fueris}{ubi + subjunctive indirect question dependent on ignorare. Same for convocaveris, ceperis, arbitraris. Asyndeton}, quos convocaveris, quid \note{consilii}{partitive} ceperis, quem nostrum ignorare \note{arbitraris}{quem + indicative = direct question}?

  2. \note{O tempora, o mores!}{accusative of exclamation} Senatus haec intellegit, consul videt: hic tamen vivit. Vivit? \note{immo}{adverb} vero etiam in senatum venit, fit publici \note{consilii}{objective genitive} particeps, notat et designat \note{oculis}{ablative of means} ad \note{caedem}{ad + accusative} \note{unum}{accusative direct object of notat/designat} quemque \note{nostrum}{partitive genitive}. Nos autem fortes viri satis \note{facere}{videmur + prolative infinitive} \note{rei publicae}{dative of advantage} videmur, si \note{istius}{genitive of possession} \note{furorem}{accusative direct object of vitemus} ac tela \note{vitemus}{subjunctive future unreal condition}. Ad mortem \note{te}{accusative + oportebat + infinitive (duci)}, Catilina, \note{duci}{oportebat + accusative (te) + prolative infinitive} \note{iussu}{means} \note{consulis}{classification} iam pridem oportebat, in te conferri \note{pestem}{oportebat + accusative + infinitive (conferri)}, quam tu in nos \note{machinaris}{quam + indicative = relative clause}.

  3. An vero vir amplissumus, P. Scipio, \note{pontifex maximus}{in apposition with vir}, Ti. Gracchum mediocriter labefactantem statum \note{rei publicae}{genitive of classification} \note{privatus}{modifies vir} interfecit; Catilinam orbem terrae \note{caede atque incendiis}{ablative of means} \note{vastare}{cupientem + prolative infinitive} \note{cupientem}{participle agreeing with Catilinam replacing relative clause} nos \note{consules}{in apposition with nos} \note{perferemus}{future indicative active of perfero}? Nam illa \note{nimis}{adverb modifies antiqua} antiqua praetereo, quod C. Servilius Ahala Sp. Maelium \note{novis rebus}{studentem + dative} studentem manu sua occidit. Fuit, fuit ista \note{quondam}{adverb} in hac re publica virtus, ut viri fortes acrioribus \note{suppliciis}{ablative of means} civem perniciosum quam \note{acerbissimum}{in apposition with civem after quam} hostem \note{coercerent}{ut + subjunctive = result clause. coerceo coercere coercui coercitum}. Habemus \note{senatus}{genitive of possession} consultum in te, Catilina, vehemens et grave; non deest \note{rei publicae}{deest + dative} consilium neque auctoritas huius \note{ordinis}{genitive of possession}: nos, nos, dico aperte, consules desumus.
}{
  Just how far, I ask, Catiline, will you misuse our tolerance? For how long still will that madness of yours make fun of us? To what end will your unbridled audacity boast of itself? Hasn’t the nightly garrison on the Palatine moved you, haven’t the patrols of the city, hasn’t the fear of the people, hasn’t the agreement of all the good men, hasn’t this very fortified place for holding the senate, haven’t the faces and expressions of these men moved you? Do you not realise that your plans are exposed? Do you not see that your conspiracy, already bound up, is being held by the knowledge of all these men? Which of us do you judge to be unaware of what you did last night, and the night before, where you have been, which men you have assembled, what plan you have adopted?

  What times, what conduct! The senate knows of these things, the consul sees: yet this man lives. He lives? Or rather in truth he still comes into the senate, he is made a participant of public council, he marks and notes with his eyes each one of us for the slaughter. But we, brave men, seem to do enough for the republic, should we avoid the furor and missiles of that man. You, Catiline, ought to have been led to death long ago now by order of the consul, that destruction which you are divising against us ought to have been brought on to you.

  But in truth, Publius Scipio, a highly distinguished man, pontifex maximus, while he was a private citizen, killed Tiberius Gracchus when he was somewhat weakening the state of the republic: will we consuls put up with Catiline, wishing to lay waste to the world by means of murder and arson? For I leave out too much of those old examples, namely the fact that Gaius Servilius Ahala killed Suprius Maelius, studying the revolution, with his own hand. There was, there was once such virtue in this republic that brave men repressed a destructive citizen with more harsh punishment than the worst enemy. We have an agreement of the senate against you, Catiline, strong and serious, the republic does not lack resolution nor authority of this order: we, I say openly, we consuls are lacking.
}
