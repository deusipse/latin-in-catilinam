\section{Chapter VIII}
\translation{
  19. Haec si tecum, ita ut dixi, patria loquatur, nonne impetrare debeat, etiamsi vim adhibere non possit? Quid, quod tu te ipse in custodiam dedisti, quod vitandae suspicionis causa ad M. Lepidum te habitare velle dixisti? A quo non receptus etiam ad me venire ausus es atque, ut domi meae te adservarem, rogasti. Cum a me quoque id responsum tulisses, me nullo modo posse isdem parietibus tuto esse tecum, qui magno in periculo essem, quod isdem moenibus contineremur, ad Q. Metellum praetorem venisti. A quo repudiatus ad sodalem tuum, virum optumum, M. Metellum, demigrasti; quem tu videlicet et ad custodiendum diligentissimum et ad suspicandum sagacissimum et ad vindicandum fortissimum fore putasti. Sed quam longe videtur a carcere atque a vinculis abesse debere, qui se ipse iam dignum custodia iudicarit?

  20. Quae cum ita sint, Catilina, dubitas, si emori aequo animo non potes, abire in aliquas terras et vitam istam multis suppliciis iustis debitisque ereptam fugae solitudinique mandare? `Refer' inquis `ad senatum'; id enim postulas et, si hic ordo sibi placere decreverit te ire in exilium, optemperaturum te esse dicis. Non referam, id quod abhorret a meis moribus, et tamen faciam, ut intellegas, quid hi de te sentiant. Egredere ex urbe, Catilina, libera rem publicam metu, in exilium, si hanc vocem exspectas, proficiscere. Quid est, Catilina? ecquid attendis, ecquid animadvertis horum silentium? Patiuntur, tacent. Quid exspectas auctoritatem loquentium, quorum voluntatem tacitorum perspicis?

  21. At si hoc idem huic adulescenti optimo, P. Sestio, si fortissimo viro, M. Marcello, dixissem, iam mihi consuli hoc ipso in templo iure optimo senatus vim et manus intulisset. De te autem, Catilina, cum quiescunt, probant, cum patiuntur, decernunt, cum tacent, clamant, neque hi solum, quorum tibi auctoritas est videlicet cara, vita vilissima, sed etiam illi equites Romani, honestissimi atque optimi viri, ceterique fortissimi cives, qui circumstant senatum, quorum tu et frequentiam videre et studia perspicere et voces paulo ante exaudire potuisti. Quorum ego vix abs te iam diu manus ac tela contineo, eosdem facile adducam, ut te haec, quae vastare iam pridem studes, relinquentem usque ad portas prosequantur. 
}{
  If the fatherland were to speak these things with you, as I have said, surely it ought to obtain its request, even if it cannot enforce it? What about the fact that you yourself gave yourself into custody? What of the fact that for the purpose of avoiding suspicion, you were willing to live at the house of Manius Lepidus. When you weren’t received by him, you even dared to come to me, and you asked that I keep you at my house. When you also received that reply from me, namely that I in no way could be safe within the same walls (house) with you, since I was in great danger because I was in the same city as you, you came to Quintus Metellus the praetor. After you were rejected by him, you moved on to your companion, an excellent man, Marcus Metellus, whom you no doubt thought would be most diligent in guarding you and very shrewd in keeping an eye on you, and very resolute in punishing you. But how far does he seem to be owing to be absent from prison and chains, who himself has judged himself now worthy of custody?

  Under these circumstances, Catiline, do you hesitate, to go off into some other lands and entrust that life of yours, snatched away from many punishments justly due, to flight and solitude (lonely exile), if you cannot die with a calm mind? `Refer it to the senate,' you say; for you demand that, if this institution decrees it to be pleasing to itself  (it has decided) that you go into exile, you say that you will obey it. I will not put it to the senate, because it is contrary to my principles, and yet I will make you realise, what these men feel about you. Leave the city, Catiline, free the republic from fear, set out into exile, if you are waiting for this word. What is it, Catiline? Do you pay attention to anything, do you notice anything of these silent men? They acquiesce, they are silent. Why are you waiting for the authority of them speaking, the wish of whom silent you perceive?

  But if I had said this same thing to this excellent young man, Publius Sestius, if I had said it to a very brave man, Marcus Marcellus, already the senate would have laid violent hands upon me as consul in this temple with excellent justification. But because of you, Catiline, when they are resting, they are approving, when they acquiesce, they are voting, when they are silent, they are shouting, not only these men, whose authority is clearly dear to you, whose lives are very worthless, but also those Roman horsemen, the most honest and excellent men, and the rest of the very brave citizens, who stand around the senate, whose great numbers you could see, whose eagerness you could perceive, and whose voices you could hear clearly a little before. I scarcely contain the hands and missiles of those men from you for a long time now, I will easily persuade those same men to escort you leaving behind this scene, which you are long since eager to lay waste to, all the way to the gates.
}
